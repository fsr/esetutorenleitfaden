\documentclass[a4paper,12pt]{report}
\usepackage[utf8]{inputenc}
\usepackage[T1]{fontenc}
\usepackage[ngerman]{babel}
\usepackage[parfill]{parskip}
%\usepackage{eurosans}
\usepackage[top=3cm, left=3cm]{geometry}
\usepackage{setspace}
\usepackage{mdwlist}
\usepackage{graphicx}
\usepackage{eurosym}
\usepackage{tikzsymbols}
\renewcommand*\familydefault{\sfdefault}

\setcounter{secnumdepth}{-1}
\setcounter{tocdepth}{1}
\begin{document}

\title{\textbf{Leitfaden für ESE-Tutoren 2015}\\}
\date{}
\author{von\\Thomas Heinze, Berit Lochner, Denis Stein (2008), \\Marcus Hähnel und Nico Hoffmann (2009), \\Marius Melzer (2010), \\Robert Schädel (2011),\\Sascha Peukert (2013), \\Kilian Költzsch und Marc Satkowski (2014), \\Philipp Heisig und Katja Linnemann (2015)}
\maketitle
\tableofcontents
\chapter{Hinweise für Tutoren}
\section{Ansprechpartner}
Katja (linnemann@ifsr.de): 01525/2879175\\
Philipp (heisig@ifsr.de): 0176/31101907 \\
FSR/ESE-Orga (ese-orga@ifsr.de): 0351/463-38226

\section{Aufgaben eines ESE-Tutors}
Als ESE-Tutor hast du die ehrenvolle Aufgabe, die Erstis an der Fakultät zu begrüßen und ihnen mit Tipps und Erfahrungen einen leichteren Einstieg ins Studium zu ermöglichen. Folgende Punkte solltest du beachten, damit das gelingt:
\begin{itemize}
	\item \textbf{Proud to be a Tutor:} Trage während der Woche zu allen ESE-Veranstaltungen (auch Abends) das ESE-Shirt und dein Namensschild.
	\item \textbf{Präsenz zeigen:} Sei während der ESE so oft wie möglich da, komme zum Frühstück usw. mit den Erstis ins Gespräch.
	\item \textbf{Have you met Ted:} Siehst du einen einsamen Ersti, hilf ihm, Anschluss zu finden (und hab Spaß dabei \Winkey ).
	\item \textbf{Sei hilfsbereit:} Ein Ersti fragt dich etwas oder ein Ersti guckt sich verloren um? Setze alles daran zu helfen!
\end{itemize}
\section{Anwesenheit in der ESE}
Eure Anwesenheit ist zu folgenden Zeiten ausdrücklich erforderlich:
\begin{itemize*}
	\item Aufgabenbereiche in denen ihr explizit als Helfer eingetragen seid
	\item Frühstück, Tütentragen und Begrüßung am Montag (ab 9 Uhr da sein)
	\item Einschreibung (Dienstag 8:30 Uhr, Treff vorm FRZ)
	\item Schnitzeljagd (Mittwoch 12:15 Uhr)
	\item ESE-Spiel (Donnerstag 12:15 Uhr)
\end{itemize*}
Während der Vorträge am Mittwoch und der Professorenvorstellung am Donnerstag müsst ihr natürlich nicht die ganze Zeit anwesend sein, an beiden Tagen könnt ihr in der Zeit auch gut Mittag essen gehen, damit ihr 12:15 pünktlich zur Vorbereitung der Schnitzeljagd/des ESE-Spiels da seid.\\
Die Anwesenheit zu allen anderen Veranstaltungen ist wünschenswert!

\section{Scheine}
Für die Teilnahme als Tutor an der ESE kann es einen Schein über 30h/1LP für die Allgemeine Basisqualifikation (Bachelor/Master) bzw. die Berufsspezifische Schlüsselkompetenz (Diplom) geben.\\
Bedingungen für die Vergabe eines Scheines:
\begin{itemize*}
	\item an Tutorenschulung teilgenommen (Ausschlusskriterium!)
	\item ein Tutorium geleitet (Ausnahmen nach Absprache möglich)
	\item Einschreibung, Schnitzeljagd und ESE-Spiel mit betreut
	\item min. eine weitere Aufgabe organisiert oder als Tutor betreut
	\item aktive Teilnahme an der Mehrheit der Veranstaltungen in der ESE-Woche
\end{itemize*}
Uns ist klar, dass es schwierig ist, allen Tutoren wirklich 30 Stunden Arbeit zuzuteilen -- trotzdem muss euer Einsatz erkennbar sein, damit wir euch einen Schein geben können. Sonst kann es ganz schnell passieren, dass wir in Zukunft keine Scheine mehr genehmigt bekommen...\\
Also: Wenn du einen Schein haben willst, sei zu allen wichtigen Veranstaltungen da und bring dich ein.

\section{Über das Tutorium}
Ziel des Tutoriums ist Vermittlung von Informationen rund um das Studium. Der Inhalt des zweiten Teils dieser Handreichung ist das Minimum, was ihr in den Tutorien vermitteln sollt, ihr könnt die Stichpunkte gerne noch mit eigenen Einfällen ergänzen.\\
Beachtet dabei:
\begin{itemize*}
\item Die Informationen sollten möglichst \textbf{unparteiisch} und \textbf{nicht wertend} vermittelt werden.
Insbesondere sollte man vermeiden, den Erstis schon vorab Angst vor bestimmten Vorlesungen oder Dozenten zu machen oder sie zum Nichtbesuchen der Vorlesungen zu animieren. Das betrifft auch das ESE-Spiel.
\item Eine Tutoriengruppe besteht aus zwei Tutoren und ca. 15-25 Erstis.
\item Falls keiner der beiden Tutoren zu einem Thema eine Auskunft geben kann, verweist am besten auf erfahrenere ESE-Tutoren oder den FSR, anstatt (möglicherweise falsche) Spekulationen zu äußern. Montag Nachmittag wird das FSR-Büro besetzt sein, sodass ihr Leute mit spezifischeren Fragen auf Montag Nachmittag oder die Seminartutorien am Dienstag verweisen könnt.
\item Wie in den letzten Jahren hat jede Gruppe einen Namenspatron. Anhand dessen werden euch die Studenten nach der Begrüßung per Los zugeteilt.
\end{itemize*}

\section{Vor dem Tutorium zu erledigende Dinge}
\begin{itemize*}
\item Lest euch diesen Leitfaden schon mal im Ganzen durch. 
Es wäre schlecht, wenn ihr das erst im Tutorium selbst tun müsst! 
Markiert euch eventuell wichtige Punkte.
Wenn ihr Fragen habt, stellt diese beim Tutorentreffen oder per Mail.
\item Überlegt euch zusammen mit eurem Tutoriumspartner, wie ihr die Informationen vortragen wollt.
Möglicherweise wollt ihr bestimmte Dinge an die Tafel schreiben.
Vielleicht ist es am sinnvollsten, die Stichpunkte abwechselnd vorzutragen, damit der jeweils andere sich schon mal Gedanken zum nächsten machen kann.
\item Schaut in der unten stehenden Tabelle nach, wo euer Tutorium stattfindet. Den Raum bitte vorher schon mal suchen, falls ihr nicht sicher wisst, wo er sich befindet.
\item Am ESE-Montag bitte spätestens um 9:00 da sein und mit helfen (10:00 geht die Begrüßung mit den anschließenden Tutorien los)!
Wir treffen uns in der APB/E023 zum Frühstück.
\end{itemize*}

\begin{center}
\vspace{1cm}
\begin{tabular}[h]{|l|l|l|l|}
	\hline
	\textbf{Namenspatron} & \textbf{Tutor(en)} & \textbf{Raum}& \textbf{Einschreibezeit}\\ \hline
	Edsger W. Dijkstra & & APB/E005 & 9:00 Uhr\\
	Kurt Gödel & & APB/E006 & 9:10 Uhr\\
	Konrad Zuse & & APB/E007 & 9:20 Uhr\\
	Tim Berners-Lee & Kilian \& Philipp & APB/E008 & 9:30 Uhr\\
	John von Neumann & & APB/E009 & 9:40 Uhr\\
	Donald Ervin Knuth & & APB/E010 & 9:50 Uhr\\
	Alan Turing & & SCH/A214 & 10:00 Uhr\\
	Ada Lovelace & & SCH/A252 & 10:10 Uhr\\
	Grace Hopper & & SCH/A185 & 10:20 Uhr\\
	Richard Stallman & & SCH/A184 & 10:30 Uhr\\
	Linus Torvalds & & SCH/A419 & 10:40 Uhr\\
	Noam Chomsky & & MER/03 & 10:50 Uhr\\
	Christiane Floyd & & MER/01 & 11:00 Uhr\\
	Stephen A. Cook & & GER/39 & 11:10 Uhr\\
	Ken Thompson & & GER/09 & 11:20 Uhr\\
	Marc Andreessen & & GER/54 & 11:30 Uhr\\
	\hline
\end{tabular}
\end{center}

\chapter{In den Tutorien zu vermittelnde Informationen}

\section{Einführung}
\begin{itemize*}
\item Wenn ihr ausländische Studierende in eurer Gruppe habt, schickt sie bitte in den APB 1004 (Ratsaal). Das Tutorium ist von 11:10 – 13:30\\
\item Macht eine kleine Vorstellungsrunde, um euch kennenzulernen und die Atmosphäre ein bisschen aufzulockern.
Die Tutoren beginnen.
Am besten ihr erzählt, wie ihr heißt, wo ihr ursprünglich herkommt, was euch an der (Medien-) Informatik gefällt.
Die Studenten könnten zusätzlich noch erwähnen, was sie vorher gemacht haben und wieso sie sich für das Inf/MInf-Studium in Dresden entschieden haben.
Schreibt am besten an die Tafel, was ihr gerne von den Erstis wissen möchtet.
\item Falls ihr eurer Gruppe anbieten wollt, auch nach dem Tutorium eventuell aufkommende Fragen zu beantworten, schreibt eure Emailadressen an die Tafel.
\item Erzählt etwas zu eurem Namenspatron. Informationen zu diesem findet ihr im Anhang.
\end{itemize*}

\section{ESE-Woche}
\begin{itemize*}
\item ESE-Website: http://ese.ifsr.de (mit aktuellem Ablaufplan der ESE-Woche (da er sich auch in der Woche noch verändern kann), Weblinks, etc.)
\item Den Zeitplan gibt es auch im iCal Format (ICS Datei) auf ese.ifsr.de zum Download.
Kann man sich direkt in den Kalender importieren.
\item Der Ablaufplan der ESE wird am FSR-Büro hinter der Wendeltreppe hängen und auf der ESE-Webseite stehen.
\item Die wichtigen Dinge der ESE-Tüte durchgehen: NoPanic, etc.
\item Geht die ESE-Woche durch und sagt kurz etwas zu den wichtigsten Programmpunkten (siehe Zeitplan und folgende Anmerkungen zu den einzelnen Tagen)
\item Bitte betont nochmal, dass die ESE sehr gut geeignet ist, um Kommilitonen kennenzulernen.
\end{itemize*}
\vspace{0.5cm}
Ort der Begrüßung ist der Hörsaal POT/81 (Pothoff-Bau) und der der Vorträge ist die ganze Woche über die APB/E023.

\includegraphics[width=\linewidth]{./zeitplan_2015.pdf}

\subsection{Montag}
\begin{itemize*}
\item 09:00 - 09:45 Frühstück
\item 10:00 - 11:00 Begrüßung im POT/0081
\item 11:00 - 12:00 Tutorien (siehe Tabelle auf S.4)
\item ab 12:00 Mittagspause
  \small{\textit{
  \begin{itemize*}
    \item Es wäre nett den Erstis einen gemeinsamen Mensabesuch anzubieten
    \item Mensakarten gibt es immer beim Frühstück oder nachher im FSR-Büro
    \item E-Meal aufladen erklären! Automaten, Infopoint, an der Kasse (Zeitersparnis)
    \item Am Nachmittag Sprechstunde im FSR-Büro
  \end{itemize*}
  }}
\item ab 18:00 Kennlern-Spieleabend mit Grillen (Fakultät Informatik)
\end{itemize*}

\subsection{Dienstag}
\begin{itemize*}
\item 09:00 - 12:30 Frühstück mit Entertainment, Vorträge, parallel Einschreibung
  \small{\textit{
  \begin{itemize*}
    \item Wichtig! Siehe nachfolgender Text zur Einschreibung
    \item Die Vorträge sind zu Studentischer Mitbestimmung und Auslandsstudium und werden als Block jeweils 2x angeboten (9:30 und 10:30)
  \end{itemize*}
  }}
\item 12:30 - 13:00 Mittagspause
\item ab 13:00 Erstes Seminargruppentreffen
  \small{\textit{
  \begin{itemize*}
    \item Wichtig! siehe nachfolgender Text zu Seminargruppen
    \item jeweils 13:00 und 15:00 je nach Studenplan/Seminargruppe
  \end{itemize*}
  }}
\item ab 20:00 Clubwanderung (Startet beim Studentenclub CountDown (Güntzstraße 22))
\end{itemize*}


\subsection{Mittwoch}
\begin{itemize*}
\item 09:00 - 09:45 Frühstück
\item 10:00 - 10:45 Vortrag Studienangelegenheiten und Organisation sowie TUDIAS
\item ab 11:00 Taschenausgabe der Uni am Chemiebau
\begin{itemize*}
  \item \small{\textit{die Vorträge enden so, dass die Erstis genug Zeit haben die Taschen zu ergattern}}
\end{itemize*}
\item 12:00 - 13:00 Mittagspause
\item 13:00 - 16:00 Campus-Schnitzeljagd
\begin{itemize*}
  \item \small{\textit{Tutorentreff Schnitzeljagd um 12:15 Uhr}}
\end{itemize*}
\end{itemize*}

\subsection{Donnerstag}
\begin{itemize*}
\item 09:00 - 10:00 Frühstück
\item 10:00 - 12:00 Vorstellung der Professoren
\item 12:00 - 13:00 Mittagspause
\item 13:00 - 16:00 ESE-Spiel
\begin{itemize*}
  \item \small{\textit{Tutorentreff ESE-Spiel um 12:15 Uhr}}
\end{itemize*}
\item ab 19:00 ESE-Kino im KIK (Kino im Kasten)
\begin{itemize*}
	\item Film wird noch nicht verraten
	\item Eintritt für Erstis \EUR{2}
	\item zusammen mit den Mathe- und Physik-Erstis
	\item danach geht's in den Traumtänzer oder in die WU5 (wer will)
\end{itemize*}
\end{itemize*}

\subsection{Freitag}
\begin{itemize*}
\item 13:00 - 15:00 Stadtführung
\begin{itemize*}
  \item \small{\textit{bitte erfragt grob das Interesse und gebt es weiter an Franz <franz@ifsr.de>}}
\end{itemize*}
\end{itemize*}

\subsection{Samstag}
\begin{itemize*}
\item 09:00 - ca. 16:00 Wanderung in der sächsischen Schweiz
\begin{itemize*}
  \item \small{\textit{bitte erfragt grob das Interesse und gebt es weiter an Lars <lars@ifsr.de>}}
\end{itemize*}
\end{itemize*}

\subsection{Einschreibung}
\begin{itemize*}
	\item \textit{Treffen der Tutoren am Dienstag um 08:30 Uhr am Eingang zum Rechenzentrum (alle ausgenommen Frühstücksteam)!}
	\item Jede Gruppe hat zugewiesenen Zeitpunkt. Wichtig: Uhrzeit für euer Gruppe ansagen (Siehe Tabelle auf S. 2)!
	\item Empfehlung für Erstis: früh da sein, Einschreibung kann auch schneller gehen. Außerdem gibt es wichtige Vorträge und Frühstück mit Entertainment!
	\item Zettel mit Namenspatron unbedingt bis dahin behalten, muss zur Einschreibung vorgezeigt werden (ohne Zettel muss man bis zum Ende der Einschreibung warten)
	\item Sollten sich Master in die Tutorien verirrt haben:
	Sie brauchen sich nicht in Seminargruppen einzuschreiben.
	Das gleiche gilt für Lehrämtler und höhersemestrige Quereinsteiger.
	\item Mitzubringen sind:
		\begin{itemize*}
		\item Zettel mit Namenspatron
		\item Studentenausweis
		\item kompletter Semesterbogen (insbesondere wegen Login und Passwort)
		\item Wunschstundenplan (vorher raussuchen, sonst kein Einlass!)
		\item zu beachten ist, dass die Einschreiben nur von der Informatik-Fakultät getan werden kann, sie wird erst am Mittwoch für außerhalb freigeschalten.
	\end{itemize*}
	\item während der Einschreibung Eintragen in die vier folgenden Mailinglisten/Mailverteiler möglich (bei dreien ist es ratsam, bei der, auf der Job-Angebote erscheinen, sollte jeder selber wissen ob er das braucht):
		\begin{itemize*}
		\item \textbf{FSR-info} -- Verteiler des FSR über den regelmäßig Informationen zu den Vorgängen an der Fakultät und den Sitzungen des FSR kommen
		\item \textbf{inf15-info} -- Verteiler für Informationen, die konkret den Studiengang der 2015 immatrikulierten Studenten betreffen (erfahrungsgemäß \textit{sehr sehr} wenig Content)
		\item \textbf{inf15-discuss} -- Mailingliste zur Diskussion untereinander
		\item \textbf{extern} -- Jobangebote werden über diese Mailingliste verbreitet
		\item und diverse andere, siehe \\ https://www.ifsr.de/fsr:news:quicklinks\_fuer\_die\_einschreibung
	\end{itemize*}
\end{itemize*}

\subsection{Seminargruppentreffen}
\begin{itemize*}
	\item Allgemeiner Sinn der Seminargruppen: Unterstützung der Erstis, Gruppenbildung, Mentor als Ansprechpartner bei Problemen und Vermittlung aller relevanten Informationen zum Studiengang
	\item \glqq Als Einzelgänger kommt man im Studium nicht weit\grqq
	\item Bitte betonen:
	Seminargruppentreffen vermitteln wichtige Informationen, darum sollte man an allen teilnehmen!
	\item Erstes Seminargruppentreffen: nach der Einschreibung am Dienstag um 13 Uhr bzw. 15 Uhr (je nach Seminargruppe - genauer Termin wird bei der Einschreibung dann mitgegeben).
\end{itemize*}

\section{Studium}

\paragraph{Aufbau und Ablauf}
\begin{itemize}
	\item Alle wichtigen Infos: Prüfungs- und Studienordnung sowie Studienablaufplan und Modulbeschreibungen findet ihr auf der ESE-Webseite unter Infos
	\item studiengangsspezifischen Informationen im ersten Seminargruppentreffen
	\item Studiums besteht aus Modulen, Module können Vorlesungen, Übungen, Praktika und Seminare beinhalten:\\\\
	\includegraphics[width=\linewidth]{./modul.pdf}
	\begin{center}
	Die Zeichnung am besten auf die Tafel malen oder was eigenes ausdenken.
	\end{center}
	\item Viele Module: Nur eine Lehrveranstaltung (Vorlesung + Übung)
	\item Jedes Modul hat eine ausgeschriebene Anzahl an Leistungspunkten (LP).\\
	1LP = 30h \glqq Arbeitsbelastung\grqq\ (über das Semester verteilt) bestehend aus:
	\begin{itemize*}
		\item Präsenzzeit (Semesterwochenstunden, SWS)
		\item Vor- und Nachbereitung der Lehrveranstaltung
		\item Vorbereitung auf die Prüfung
		\item Prüfung selbst
	\end{itemize*}
	Leistungspunkte werden einem erst nach bestandener Modulprüfung angerechnet!
\end{itemize}

\paragraph{Vorlesung, Übung, Praktikas}
\begin{itemize}
	\item \textbf{Vorlesung} wird von einem Dozenten, meist einem Professor, gehalten\\
	Ort meistens Hörsaalzentrum (HSZ)
	\item fast alle Vorlesungen: zugehörige \textbf{Übungen} (Übungsleiter normalerweise nicht Professor sondern Mitarbeiter oder Studenten höherer Semester)
	\item Einschreibung in Stundenpläne = Einschreibung in Seminargruppen $\rightarrow$ eure Seminargruppe ist eure Übungsgruppe
	\item Aufgaben in Übungen entsprechen meist Aufgaben, wie sie in den Prüfungen zu erwarten sind
	\item Angedacht ist: Übungsaufgaben (meist ein A4-Zettel) vor der tatsächlichen Übung lösen $\rightarrow$ in Übung werden dann Ergebnisse besprochen und Fragen geklärt\\
	$\longrightarrow$ machen viele nicht, ist aber wirklich sinnvoll und die beste Prüfungsvorbereitung!!
	\item \textbf{Praktika} = meist uniinterne, praktische Lernveranstaltungen\\
		entweder während der Vorlesungszeit oder als zusammenhängender Block in der vorlesungsfreien Zeit
\end{itemize}
		
\paragraph{Prüfungen}
\begin{itemize}
	\item bilden Abschluss einer Lehrveranstaltung
	\item im Normalfall in der Kernprüfungszeit (ersten 4-5 Wochen der vorlesungsfreien Zeit)
	\item Jedes Modul hat Modulnote:\\
		kann aus verschiedenen Prüfungsleistungen bestehen, dann nicht alle davon zwingend für das Bestehen des Moduls nötig, denn:\\
		mehrere Prüfungsleistungen $\rightarrow$ oft arithmetisches Mittel für Berechnung der Modulnote (Ausnahme z.B. Mathe: Wichtung)\\
		$\longrightarrow$ nichtbestandene Prüfungen können mit anderer Note ausgeglichen werden		
	\item Theoretisch können eine oder mehrere bestandene Prüfungsvorleistungen (PVL) für Zulassung zu einer Prüfungsleistung nötig sein\\
		\textbf{Beispiel} Mathe: Übungen abgeben $\rightarrow$ 50\% der Übungspunkte müssen erworben sein, um an der zweiten Prüfung teilzunehmen
	\item Mathe-Modul im ersten Semester: \glqq Nikolausklausur\grqq\ in der Mitte des Semesters (erste Prüfungsleistung, zusätzlich zu der am Ende des Semesters)\\
	\textbf{Hinweis:} Mathe im ersten Semester viel Stoff und nicht einfach $\rightarrow$ nicht schleifen lassen!
	\item Prüfungen können zwei mal wiederholt werden
	\begin{itemize*}
		\item 1. Wiederholung: ein Jahr (zwei Semester) Zeit
		\item 2. Wiederholung: nochmals ein Semester Zeit
	\end{itemize*}
	2. Wiederholungsprüfung nicht geschafft $\rightarrow$ Exmatrikulation, Studium nicht geschafft
	\item wieder abmelden von einer Prüfung: ohne Angabe von Gründen bis 3 Werktage (bzw. 2 Wochen bei mündl. Prüfungen) vor dem Prüfungstermin möglich
	\item nach Ende dieser Frist: Rücktritt - nur bei Krankheit o.ä. zulässig\\ 
		muss Prüfungsamt unverzüglich schriftlich mitgeteilt werden (ärztliches Attest o.ä.)
	\item Freiversuchsregelung: siehe \S15 der Prüfungsordnung (Laut Hochschulgesetz abgeschafft, aber da sie noch in der Ordnung steht, gilt die Regelung noch!) - kann genutzt werden, wenn man Prüfung eher, als im Ablaufplan vorgesehen, schreibt $\rightarrow$ nach Nichtbestehen hat man immer noch Erstversuch, auch Note verbessern ist (innerhalb eines Semesters) möglich (immer rechtzeitig beim Prüfungsamt melden!)
	\item Prüfungsausschuss (je Studiengang): bearbeitet Anträge für Anerkennung bzw. Anrechnung von Studien- und Prüfungsleistungen, Anträge auf Prüfungsfristverlängerung, Anträge auf Annullierung einer Prüfung, etc.
	\item Regelstudienzeit überziehen: sowohl bei Bachelor als auch bei Diplom bis zu 4 Semester, danach Abschlussprüfung das erste Mal nicht bestanden, noch 1 Jahr Zeit für Wiederholung
	\item Studienzeit kann durch Uraubs- und Gremiensemester verlängert werden:
	\begin{itemize*}
		\item Urlaubssemester: studienfreies Semester, in dem neuerdings auch Prüfungen geschrieben werden können (valider Grund für Urlaubssemester nötig!)
		\item Gremiensemester: max. 3, Reduzierung der Fachsemesterzahl, kann man durch engagierte Gremienarbeit (Fachschaftsrat, Fakultätsrat, Prüfungsausschuss, etc.) bekommen
	\end{itemize*}
\end{itemize}

\textbf{Einschreibung in Vorlesungen, Übungen und Prüfungen, Abrufen der Prüfungsergebnisse: }elektronische Einschreibesystem jExam (http://jexam.de)\\\\
Eine Zusammenstellung von fürs Studium wichtigen Dokumenten und Links zu vielen Skripten findet man auch auf der FSR-Seite (ifsr.de $\rightarrow$ Studium).
Diplomer können aufgrund der ähnlichen Lehrveranstaltungen vorerst im Bereich Bachelor nach Skripten gucken, der Bereich \glqq Altes Diplom\grqq\ ist für das \glqq alte\grqq\ Diplom, nicht das neue.

\section{Drucken, Kopieren, Rechentechnik}
\begin{itemize}
	\item ZIH-Login: gilt sowohl fürs Rechenzentrum als auch für jExam, Email und WLAN\\
			Username: \glqq sNr\grqq (auf Immatrikulationsbögen)
			Passwort: für den Erstlogin Imma-Bogen, ist aber dringendst abzuändern, da jegliche Funktionalität sonst stark eingeschränkt ist
  \item \textbf{Deshalb wichtig!} Falls noch nicht geschehen: Passwort noch heute im IDM ändern, damit morgen die Einschreibung funktioniert: https://idm-service.tu-dresden.de -- ruhig als Hausaufgabe aufgeben.
	\item Jeder Student hat Exchange-postfach beim ZIH
	\begin{itemize*}
		\item Adresse: sNR@msx.tu-dresden.de, namensbasierte Weiterleitung \\(vorname.nachname[Zahl]@mailbox.tu-dresden.de)
		\item Abrufen: Outlook/Exchange, IMAP, \\Webinterface unter https://msx.tu-dresden.de
		\item \textbf{Wichtig:} regelmäßig abzurufen oder an andere Adresse weiterleiten lassen\\
		$\rightarrow$ Informationen über Prüfungsanmeldung, Rückmeldung zum kommenden Semester etc.
	\end{itemize*}
	\item zwei WLAN-Netze auf dem Uni-Gelände:
	\begin{itemize*}
		\item VPN/Web: einfacher einzurichten, offenes Netzwerk\\
		$\rightarrow$ nach dem Aufrufen der ersten Website Login-Daten im Browser eingeben
		\item eduroam: auch an anderen Universitäten (sogar international) verwendet\\
		Einrichten: Anleitung auf der Website der TU Dresden, falls notwendig benötigtes Zertifikat herunterladen\\
		- Linux: funktioniert mittlerweile mit allen wichtigen Networkmanagern (Gnome Network Manager, KNetworkmanager, wicd, auf der Konsole per wpasupplicant...)\\
		- Windows/OS X: einfach mit ZIH Login anmelden
	\end{itemize*}
	\item Rechenzentrum:
	\begin{itemize*}
		\item Computer-Arbeitsplätze (meist Dualboot mit Windows und Linux (Ubuntu)) und Wlan-Arbeitsplätze (Monitore mit VGA-Eingang, an die man sein Notebook anschließen kann)
		\item Spezialräume mit Multimedia-Equipment
		\item Technik-Ausleihe (z.B. für Kameras, Beamer, etc.)
	\end{itemize*}
	\item Drucken und Kopieren: 
	\begin{itemize*}
		\item im FSR-Büro (ab 2ct/Seite)
		\item an unterschiedlichen Standorten in der Uni mit beim Stura zu erwerbender Kopierkarte (3,7-5ct/Seite)
		\item in der SLUB (5-15ct/Seite)
		\item an den diversen Copyshops auf dem Unigelände
	\end{itemize*}
	\item MS DreamSpark: -- Verweis auf Kram in den Tüten: Zum Beispiel die Blöcke sind von DreamSpark gesponsert
	\begin{itemize*}
		\item kostenlose Microsoft Software für Studenten (Windows, Visual Studio,...)
		\item seit diesem Semester ohne Registrierung einfach mit ZIH-Login nutzbar
	\end{itemize*}
\end{itemize}

\section{Studentische Selbstverwaltung}

\subsection{Fachschaftsrat}
\begin{itemize}
	\item Vertretung der Studenten auf Fakultätsebene
	\item besteht derzeit aus 17 Mitgliedern
	\item wird immer im Wintersemester für ein Jahr neu gewählt
	\item Ansprechpartner bei Fragen und Problemen
	\item veranstaltet die ESE, Professorenstammtische, die Spieleabende, die Lehrevaluationen, usw.
	\item Sitzungen: jede Woche montags um 18:30 im Großen Ratssaal APB/1004, sind öffentlich, Gäste sind herzlich willkommen
	\item FTP-Server (ftp://ftp.ifsr.de): Protokolle, Klausuren vergangener Jahre, Komplexprüfungsprotokolle (fürs Hauptstudium / Master)
	\item Website ifsr.de: soll in naher Zukunft überarbeitet werden, trotzdem gibt es regelmäßige Infos zu Spieleabenden, zum Studium usw.
	\item Büro: APB/E017 (hinter der Treppe, neben dem Cafe ascii)
	\item Auch Erstis können sich gern zur nächsten Wahl für den FSR aufstellen lassen! Alternativ suchen wir für Ende November wieder viele Wahlhelfer!
\end{itemize}

\subsection{Studentenrat}
\begin{itemize}
	\item kurz StuRa, Vertretung der Studenten auf Uniebene
	\item vertritt studentische Interessen gegenüber der Universitätsleitung
	\item verteilt Gelder
	\item Beratungsangebote u.a. zu BAföG, Sozialem und bei Rechtsfragen/-problemen
	\item Mitglieder sind Entsandte aus den FSRen der einzelnen Fakultäten
	\item Sitzungen: zweiwöchentlich an Donnerstagen ab 19:30 Uhr in der StuRa-Baracke, ebenfalls öffentlich
\end{itemize}

\section{Studentisches Leben}

\subsection{Studentenclubs}
\begin{itemize}
	\item 15 Stück in Dresden $\rightarrow$ damit kann man Dresden durchaus Hauptstadt der Studentenclubs nennen
	\item ehrenamtlich von Studenten geführt
	\item  Auflistung z.B. unter http://www.vdsc.de (Vereinigung Dresdner Studentenclubs)
	\item Club, der zur Fakultät Informatik \textit{gehört}: \glqq CountDown\grqq\ -- kurz: das CD (Nähe Straßburger Platz)\\
	$\rightarrow$ Ausgangspunkt Clubwanderung am Dienstag
	\begin{itemize}
		\item früher "Club Dürerstraße", befand sich direkt in der alten Fakultät Informatik und war tagsüber Cafe
		\item nach Neubau: Cafe ist mit umgezogen: Ausgründung ascii, Club ist auf die Güntzstraße gezogen
	\end{itemize}
\end{itemize}

\subsection{Kino}
\begin{itemize}
	\item Studentenkino \glqq Kino im Kasten\grqq\ -- kurz KiK in der Philosophischen Fakultät in der August-Bebel-Straße
	\item Infos und Programm unter http://kino-im-kasten.de
	\item dort findet am Donnerstag das ESE-Kino statt!
\end{itemize}

\section{Die Universität}
\subsection{SLUB}
\begin{itemize}
	\item Sächsischen Landes- und Universitätsbibliothek
	\item für Studenten kostenlos
	\item Bücher leihen, ruhige Räumlichkeiten zum Lernen nutzen
	\item Gruppenräume für gemeinsames Arbeiten vorhanden (ggf. reservieren)
	\item Informatikbücher befinden sich in der Lehrbuchsammlung im Haupthaus und im gegenüberliegenden \glqq DrePunct\grqq
\end{itemize}

\subsection{Mensen}
\begin{itemize}
	\item die der Fakultät nächstgelegene Mensa: Alte Mensa
	\item Emeal-Karte zum zahlen in allen Mensen und Cafeterien der TU Dresden (wird auch an einigen Orten der anderen Hochshulen in Dresden akzeptiert)
	\item erhältlich in den Mensen oder ganz einfach während der ESE beim Frühstück oder im FSR-Büro
	$\rightarrow$ Emeal-Bescheinigung (Imma-Bogen), \EUR{5} Kaution, Studentenausweis, Ausweis mitbringen
	\item Aufladen am Automaten in den großen Mensen, an der Kasse (zum Teil nur mit Bargeld möglich), auf Wunsch per Autoload
\end{itemize}

\subsection{Unisport}
Das Universitätssportzentrum (USZ) biete viele Sportarten an
\begin{itemize*}
\item \textbf{Aktuelles Semester:} wegen der Flüchtlingssituation aktuell nicht das komplette Sportangebot
\item Sportprogramm ab 01.10. auf der Homepage des USZ
\item Einschreibung WS 15/16 am 13.10. Nachmittags, gestaffelt nach Sportarten
\item Preise für Studenten recht günstig (zwischen ca. 15-40 Euro pro Semester je nach Sportart)
\item Bei begehrten Sportarten schnell sein, da viele Kurse nach wenigen Minuten voll sind
\end{itemize*}

\subsection{Sprachkurse}
\begin{itemize*}
\item Bis zu 10 SWS kostenlose Sprachkurse beim Sprachenzentrum LSK für jeden Studenten
\item Je nach Kurs gibt es Sprachzertifikate
\item Für den Informatik-Master wird Englisch benötigt
\item Webseite: http://lskonline.tu-dresden.de
\end{itemize*}

\subsection{Nebenjobs}
\begin{itemize*}
\item \textbf{SHKs (Studentische Hilfskräfte)} werden überall an der Uni gesucht (gerade Informatiker)
\item Informatikbezogene Stellenangebote finden sich im APB per Aushang oder auf der Fakultätswebseite
\item Es gibt die STAV e.V. (Studentische Arbeitsvermittlung) für weitere Nebenjobs
\end{itemize*}

\section{Wohnen in Dresden}
\begin{itemize*}
  \item Hauptwohnsitz nach Dresden $\rightarrow$ \glqq Begrüßungsgeld\grqq\ von 150 Euro, beantragen beim Studentenwerk
  \item Zweitwohnungssteuer in Dresden seit 2006 fällig. Bewohner einer WG oder eines Wohnheims kann Widerspruch einlegen
	\item neuen Wohnsitz innerhalb von 14 Tagen melden\\
  (http://www.dresden.de/de/rathaus/ortsaemter.php).
\end{itemize*}

\section{Dresden}

\subsection{Altstadt}
\begin{itemize*}
	\item Kulturelles Zentrum: Semperoper, Zwinger, Frauenkirche, etc.
  \item \textbf{Stadtführung am Freitag empfehlen}
	\item Einkaufsmeile mit Prager Straße, Altmarktgalerie, etc.
	\item Schauspielhaus: Erstis kommen für \EUR{3,50} ins Theater, auch sonst für Studenten extrem preiswert
\end{itemize*}

\subsection{Neustadt}
\begin{itemize*}
	\item Kneipenviertel, alternatives Viertel
	\item viele Programmkinos: Schauburg, Casablanca, Thalia, etc.
	\item einmal im Jahr großes Straßenfest: BRN - Bunte Republik Neustadt
\end{itemize*}

\section{Ansprechpartner bei Problemen}
\begin{itemize*}
	\item FSR: fsr@ifsr.de
	\item Studiendekan: Prof. Weber (allgemein), Prof. Friedrich (Lehramtsstudiengänge), Prof. Hölldobler (englischsprachige Studiengänge).
	\item Studentische Studienberatung: Sascha Peukert (studienberatung-inf@ifsr.de) und Philipp Heisig (studienberatung-minf@ifsr.de) oder beide unter studienberatung@ifsr.de.
	\item studentische Vertreter im Prüfungsausschuss: Sascha Peukert, Duc Tien Nguyen.
	\item Prüfungsamt (APB/3039 und 3040).
	\item Rechtsverbindliche Auskünfte gibt es aber nur vom Prüfungsausschuss, Anträge über das Prüfungsamt stellen!
\end{itemize*}

\section{Rundgang durch die Fakultät}
Macht einen kleinen Rundgang durch die Fakultät und zeigt mindestens: ascii, FSR-Büro, E023 (Vorlesungssaal), Rechenzentrum, Prüfungsamt, Ratssaal (1004) mit Hinweis auf FSR Sitzungen
Bietet doch euren Erstis einen Besuch in der Mensa an!

\chapter{Namenspatrone}
\section*{Alan Turing (1912 - 1954)}
\begin{itemize*}
	\item Engländer
	\item schuf Großteil der theoretischen Grundlagen der modernen Informatik
	\item unterstützte Entschlüsselung der Enigma
	\item schrieb das erste Schach-Computerprogramm
	\item entwickelte ein Testverfahren, welches herausfinden kann, ob eine Maschine intelligent ist (Turing-Test)
	\item ``Oscar'' der Informatik nach ihm benannt (Turing-Preis)
	\item Studium: Turing-Maschine (TheoInf, evtl. FS)
\end{itemize*}

\section*{Edsger W. Dijkstra (1930 - 2002)}
\begin{itemize*}
	\item Niederländer
	\item Djikstra-Algorithmus zur Berechnung des kürzesten Wegs in einem Graphen
	\item Semaphore zur Synchronisation von Threads
	\item berühmt wegen Abhandlung ``Goto considered harmful''
	\item Einführung der strukturierten Programmierung (verwendet in Programmiersprachen wie Pascal oder C)
	\item Studium: Dijkstra-Algo (AuD), Semaphore (BuS)
\end{itemize*}

\section*{Kurt Gödel (1906 - 1978)}
\begin{itemize*}
	\item Deutscher
	\item Beiträge zur Relativitätstheorie und klassischen Logik
	\item viele Beiträge zur Prädikatenlogik (Vollständigkeit und Entscheidungsproblemen)
  \item Entwicklung der Gödelnummer
  \item Studium: Prädikatenlogik, Gödelnummer (TheoInf)
\end{itemize*}

\section*{Konrad Zuse (1910 - 1995)}
\begin{itemize*}
	\item Deutscher
	\item gilt als Erfinder des modernen Computers
	\item Konstruktion der Computer Z1 bis Z4
	\item Entwicklung der ersten höheren Programmiersprache Plankalkül
	\item theoretische und praktische Arbeit zur Darstellung von Gleitkommazahlen (Exponent, Mantisse)
	\item Studium: Gleitkommazahlen, Vektorrechner (RA)
\end{itemize*}

\section*{Donald Ervin Knuth (geb. 1938)}
\begin{itemize*}
	\item Amerikaner
	\item Verfasser von ``The Art of Computer Programming'' (Standardwerk über Datenstrukturen \& Algorithmen)
	\item entwickelte Satzsystem TeX
	\item Erfinder des KMP-Algorithmus (String Matching) und des Buddy-Verfahrens (Speicherverwaltung)
	\item Studium: KMP (AuD), Buddy (BuS)
\end{itemize*}

\section*{John von Neumann (1903 - 1957)}
\begin{itemize*}
	\item Österreich-Ungar
	\item Beiträge in Quantenmechanik und Spieltheorie
	\item Entwicklung der von-Neumann-Architektur
	\item entdeckte während seiner Arbeit an der von-Neumann-Architektur viele Programmkonstrukte: linked subroutines, lists, Einführung der binären Kodierung, double precision arithmetic, Mergesort etc.
	\item Studium: von-Neumann-Architektur (TGI), Binäre Kodierung, Double precision arithmetic (RA)
\end{itemize*}

\section*{Tim Berners-Lee (geb. 1955)}
\begin{itemize*}
	\item Engländer
	\item gilt als Begründer des WWW (Web-Developer, höhö)
	\item erfand HTML
	\item schrieb den ersten Browser
	\item Vorsitzender des W3C (Gremium, welches grundlegende Standards des Netzes spezifiziert)
	\item Studium: WWW (RN)
\end{itemize*}

\section*{Ada Lovelace (1815 - 1852)}
\begin{itemize*}
	\item Engländerin
	\item gilt als erste Programmiererin der Welt
	\item beschrieb, wie man die Bernoulli-Zahlen mit einer Maschine berechnen kann
	\item Namensgeberin der Programmiersprache Ada
\end{itemize*}

\section*{Grace Hopper (1906 - 1992)}
\begin{itemize*}
	\item Amerikanerin
	\item 1954 ersten Compiler (A-0) entwickelt
	\item Mitarbeit am Mark I und Mark II
	\item fand den ersten (Hardware) Bug (heißt Bug, weil es eine Motte war)
	\item Studium: Entwicklung der Rechentechnik (RA)
\end{itemize*}

\section*{Richard Stallman (geb. 1953)}
\begin{itemize*}
	\item Amerikaner
	\item Präsident der Free Software Foundation
	\item entwickelte die GPL
	\item entwickelte die ersten Versionen des gcc, gdb und von GNU Emacs
\end{itemize*}

\section*{Linus Torvalds (geb. 1969)}
\begin{itemize*}
	\item Finne
	\item Initiator des freien Kernels Linux
	\item bis heute einer der führenden Entwickler von Linux
	\item Studium: Linux (BuS)
\end{itemize*}

\section*{Noam Chomsky (geb. 1928)}
\begin{itemize*}
	\item Amerikaner
	\item Beiträge in versch. Bereichen wie Linguistik und Psychologie
	\item Chomsky-Hierarchie teilt formale Sprachen in Klassen (Typ 0 bis Typ 3) ein
	\item wichtige Grundlage der Theoretischen Informatik, speziell des Compilerbaus
	\item Studium: Chomsky-Hierarchie (FS, TheoInf)
\end{itemize*}

\section*{Christiane Floyd (geb. 1943)}
\begin{itemize*}
	\item Österreicherin
	\item erschuf die erste Entwicklungsumgebung Maestro I
	\item erste Professorin im Bereich Informatik in Deutschland
	\item umfangreiche Forschung im Bereich von Softwarentwicklungsmethoden
\end{itemize*}

\section*{Stephen A. Cook (geb. 1939)}
\begin{itemize*}
	\item Amerikaner
	\item forscht hauptsächlich im Bereich der Theoretischen Informatik
	\item formulierte in seinem berühmtesten Paper den Begriff NP-Vollständigkeit
	\item in diesem Paper lässt er die Frage offen, ob P=NP, welches eine der zentralen Fragestellung der modernen Informatik ist
	\item Studium: P=NP, NP-Vollständigkeit (TheoInf)
\end{itemize*}

\section*{Ken Thompson (geb. 1943)}
\begin{itemize*}
	\item Amerikaner
	\item entwickelte Programmiersprache B (Vorgänger von C)
	\item erschuf mit Ritchie erste Version von UNIX
	\item ebenfalls Arbeit an BS Multics und Plan9
	\item schrieb frühe Versionen von Tools wie der ersten Shell sh, die bis heute Bestandteil moderner Betriebssysteme sind
	\item Studium: Unix (BuS)
\end{itemize*}

\section*{Marc Andreessen (geb. 1971)}
\begin{itemize*}
	\item Amerikaner
	\item Mitbegründer der Netscape Corp.
	\item schrieb einen der ersten weit verbreiteten Browser Mosaic
	\item entwickelte mit am Netscape Navigator, der die Grundlage vom Mozilla Firefox ist
\end{itemize*}

\section*{Rudolf Bayer (geb. 1939)}
\begin{itemize*}
	\item Deutscher
	\item forscht im Bereich der Datenbanken
	\item entwickelte Datenstruktur B-Bäume (Grundlage der meisten modernen Datenbank- und Dateisysteme)
	\item Studium: B-Bäume (DB)
\end{itemize*}

\section*{Dennis Ritchie (1941 - 2011)}
\begin{itemize*}
	\item Amerikaner
	\item entwickelte mit Thompson Unix
	\item erschuf Programmiersprache C
	\item Studium: C (AuD), Unix (BuS)
\end{itemize*}

\section*{Claude Shannon (1916 - 2001)}
\begin{itemize*}
	\item Amerikaner
	\item beschäftigte sich mit mathematischen Grundlagen der Kommunikation
	\item Einführung von verschiedenen Begriffen/Theoremen wie Entropie und Nyquist-Shannon-Abtasttheorem
	\item Studium: Nyquist-Shannon-Abtasttheorem (RN, IKT, EMI, MMS), Entropie, Shannon-Fano-Verfahren (beides IKT, EMI, MMS)
\end{itemize*}

\section*{C.A.R. Hoare (geb. 1934)}
\begin{itemize*}
	\item Brite
	\item beschäftigte sich mit den theoretischen Grundlagen von Programmiersprachen
	\item entwickelte Hoare-Kalkül (Korrektheit von Algorithmen)
	\item erfand Quicksort
	\item Studium: Hoare-Kalkül (Prog), Quicksort (AuD)
\end{itemize*}

\section*{Alonzo Church (1903 - 1995)}
\begin{itemize*}
	\item Amerikaner
	\item Mathematiker und Logiker
	\item hat 1930 u.a. das Lambda-Kalkül erfunden, welches die Grundlage für funktionale Programmiersprachen darstellt
	\item Studium: Lambda-Kalkül (Prog)
\end{itemize*}

\end{document}
